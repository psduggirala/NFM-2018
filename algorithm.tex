\section{Aggregation and Deaggregation Strategies}
\label{sec:agdag}

One of the primary drawbacks of Algorithm~\ref{alg:algoHybrid} is in handling the discrete transitions. 
%
Suppose that in a given location, the number of stars that overlap with the guard of a discrete transition (line~\ref{ln:guardIntersection} in Algorithm~\ref{alg:algoHybrid}) is $m$.
%
As a result, the number of stars in the $queueStars$ will become $O(m^2)$ after 2 discrete. After $t$ number of discrete transitions, the number of states in $queueStars$ grows to $O(m^t)$.
%
To avoid the exponential blow up of the number of sets in $queueStars$, reachable set computation tools often use aggregation.

In aggregation, the set of all stars in $queueStars$ that are making a discrete transition to the same mode are collected together. 
%
Say, these states are $S_1, S_2, \ldots, S_m$.
%
Then, an overapproximation of these $S'$ is computed such that $S_1 \cup S_2 \cup S_3 \ldots \cup S_m \subseteq S'$. 
%
Instead of computing the reachable set for each of $S_1, S_2, \ldots, S_m$, the reachable set of $S'$ is computed in the future modes.

There are two main drawbacks of this aggregation mechanism. 
%
First, the collection of sets $S_1, S_2, \ldots, S_m$ is often a non-convex set. 
%
Whereas the representation used for computing reachable set is for convex sets. 
%
Therefore, this overapproximation of a non-convex set by a convex set is very conservative.
%
More worryingly, the reachable set of $S'$ will trigger additional discrete transitions that would not happen while computing the reachable sets using $S_1, S_2, \ldots, S_m$.
%
Such discrete transitions are artifacts of the conservative overapproximation during the aggregation process.


To overcome the above mentioned challenges, 
