\section{Introduction}
\label{sec:intro}

Aeronautical systems such as air-traffic control protocols, auto-pilot software, and satellite maneuver protocols are safety critical in nature.
%
Bugs in such systems can lead to many unsafe scenarios, causing loss of property and in some cases, life.
%
Testing such systems extensively under various scenarios often provides the system designer and the user with confidence that the systems functions in a safe manner.
%
However, such extensive testing has its own set of challenges.
%
For systems such as air-traffic control protocols, testing the protocols under various scenarios is an expensive endeavour.
%
In the case of satellite maneuver protocols, it is impossible to create a controlled test bed on earth to test such systems.
%
Moreover, once deployed, updates to the control algorithms on satellite maneuver protocols is very difficult, if not impossible.


One of the widely used method for ensuring that the system does not encounter any unsafe scenarios is model-based analysis.
%
In this method, a high-fidelity model of the system is created and extensive testing and verification is performed on the model.
%
% Model-based analysis is also helpful in reducing the number of test scenarios of the system.
%
% It is particularly well suited for satellite maneuver protocols as creating a test bed for such systems is impossible.
% 
% 
Formal verification of safety properties of several aeronautical systems modelled as a hybrid automata are widely available in the literature.
%
This is because hybrid automata is a well suited formalism for modeling satellite maneuver protocols. 
%
% The motion of satellite in each mode is modelled as a differential equation. 
%
% The change in mode of operation is modelled as a transition among these modes.
%
In this paper, we perform safety analysis of satellite rendezvous mission modelled as a hybrid automata.


Our analysis relies on computing an artifact called {\em reachable set}. 
%
Given a set of initial configurations (often uncountable) $\Theta$, the reachable set is the set of all possible configurations encountered by the system trajectories starting from $\Theta$. 
%
Since the reachable set is also an uncountable set, the reachable set computation techniques compute a symbolic representation of the reachable set.
%
Typically, symbolic representations of convex sets such as support functions, polytopes, ellipsoids, zonotopes, etc. are used.
%
This is because operations such as linear transformation, intersection, and minkowski sum can be efficiently performed using such convex representations.


These convex representations are at a significant disadvantage while performing mode switches.
%
Consider the case of a satellite rendezvous mission that involves a mode switch from continue to abort.
%
This mode switch can happen at any time in a given interval. 
%
Due to this non-determinism, the initial set of states in the abort mode is a convex overapproximation of the reachable set. 
%
Often, this overapproximation is too conservative and hence, such reachable set computation is not useful in determining whether all the behaviors of the systems are safe. 
%
\sridhar{Probably a spaceEx figure for state space exploration might come in handy here.}

This paper exclusively focuses on improving the accuracy of the reachable set by presenting various aggregation strategies for discrete transitions in linear hybrid automata.
%
We present a data structure called Aggregated Directed Acyclic Graph (AGGDAG) for keeping track of aggregations and their dependence on the reachable set.
%
Using AGGDAG, we present several deaggregation strategies based on the counterexamples generated for the safety specification. 
%
We also provide soundness and relative completeness guarantees of our deaggregation strategies.
%


This work on developing deaggregation strategies is crucial for performing safety analysis of a satellite rendezvous mission. 
%
\sridhar{Might have to be ommitted.} Due to the non-determinism in the mode switching, state of the art tools are only able to analyze restricted switching behavior of this satellite rendezvous mission.
%
We demonstrate that using the deaggregation strategies presented in this paper, we can perform accurate safety analysis very quickly.
