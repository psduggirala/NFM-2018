% !TEX root = emsoft-2019.tex
\section{Introduction}
\label{sec:intro}

Aeronautical systems such as air-traffic control protocols, auto-pilot software, and satellite maneuver protocols are safety critical in nature.
%
Design errors in such systems, such as floating point bugs in Ariane 5 spacecraft, might lead to unsafe behaviors causing loss of property and in some cases, life.
%
Testing such systems extensively under various scenarios might give confidence to the system designer that the systems functions in a safe manner.
%
However, such extensive testing is not always possible. 
%has its own set of challenges.
%
%For systems such as air-traffic control protocols, testing the protocols under various scenarios is an expensive endeavour.
%
%In the case of satellite maneuver protocols, it is impossible to create a controlled test bed on earth to test such systems.
%
%Moreover, once deployed, updating the control algorithms on satellite maneuver protocols is very difficult, if not impossible.
%
%
One of the widely used method for ensuring that the system does not encounter any unsafe scenarios in such cases is model-based analysis.
%
In this method, a high-fidelity model of the system is created and extensive testing and verification is performed on the model.
%
% Model-based analysis is also helpful in reducing the number of test scenarios of the system.
%
% It is particularly well suited for satellite maneuver protocols as creating a test bed for such systems is impossible.
% 
% 
Hybrid automata is a well suited framework for modeling such safety critical systems and formal verification approaches for proving safety properties of several aeronautical systems modeled as a hybrid automata are widely available in the literature~\cite{tomlin1998conflict,pallottino2002conflict,prabhakar2009verifying,johnson2012satellite,munoz2013tcas,zhao2014formal,duggirala2014temporal,jeannin2015formally}.
%
%This is because hybrid automata is a well suited formalism for modeling satellite maneuver protocols. 
%
% The motion of satellite in each mode is modelled as a differential equation. 
%
% The change in mode of operation is modelled as a transition among these modes.
%

In this paper, we perform safety analysis of \new{two case studies, satellite rendezvous mission and gearbox meshing system} modeled as a hybrid automata.
%
We adopt a widely used technique for establishing safety properties of hybrid systems: {\em reachable set computation}.
%
%Our analysis relies on computing an artifact called {\em reachable set}. 
%
Given a set of initial configurations $\Theta$ (often uncountable), the reachable set is the set of all possible configurations encountered by the system trajectories starting from $\Theta$. 
%
Since the reachable set is also uncountable, we compute its symbolic representation.
%
If the reachable set does not contain any unsafe configurations, then one can prove that the safety specification is satisfied.
%
Often, symbolic representations of convex sets such as polytopes~\cite{Frehse05_phaver}, zonotopes~\cite{girard2006efficient}, support functions~\cite{spaceex}, etc. are used because operations such as linear transformation, intersection, convex hull, and Minkowski sum can be easily performed over these representations.
%
%This is because operations such as linear transformation, intersection, and minkowski sum can be efficiently performed using such convex representations.

Such representations, however, are at a disadvantage while performing mode switches.
%
Due to the non-determinism of mode switching behavior, one has to compute an overapproximation of all the states that can perform the mode switch in the chosen representation.
%
This operation is called aggregation.
%
Often, this overapproximation is very conservative and might overlap with the unsafe set of configurations.
%
\new{Algorithmically selecting the sets for performing aggregation to improve accuracy involves combinatorial search and is a challenging task.}
%
%\new{It is possible to reduce this overapproximation by aggregating only a select few sets.} 
%
%\new{Performing this task algorithmically is challenging because it involves checking all possible sets to be aggregated and selecting the most appropriate aggregation that yields the best accuracy.}
%
However, if one does not perform aggregation, then one has to keep track of exponential number of symbolic representation after a finite number of mode switches.


This paper exclusively focuses on aggregation and de-aggregation strategies in reachable set computation. 
%
We present two aggregation strategies, first, template based, and second, convex hull based. 
%
For efficient de-aggregation, we introduce a data structure called Aggregated Directed Acyclic Graph (AGGDAG), and explain our de-aggregation strategy.
%
\new{We demonstrate that goal driven aggregation mechanisms along with template based aggregation and deaggregation can handle challenging case studies of satellite rendezvous and gerbox meshing involving challenging discrete transitions.}

%\new{The two case studies considered in this paper are satellite rendezvous mission and gearbox meshing system.}
%, present significant challenge in handling the discrete transitions.}
%These improvements in aggregation and de-aggregation strategies enable us to perform the safety analysis of a satellite rendezvous mission.
%
In a rendezvous mission, the satellite can operate in one of the two modes: approach or abort. 
%
In the approach mode, the trajectory of the satellite proceeds towards the rendezvous point. 
%
In the abort mode, the decision to rendezvous is aborted and hence the satellite should maintain a safe distance from the rendezvous point. 
%
%Naturally, one cannot decide to abort the rendezvous when the satellite is too close to the rendezvous point.
%
%In this paper, we establish that if the decision to abort is performed in a given interval of time, the safe distance from the rendezvous point is maintained.
%
The nondeterminism in the mode switch from approach to abort makes the safety analysis very challenging.
%
\new{Gear meshing system models the behavior of a sleeve switching gears.}
%
\new{Depending on the angular position at which the sleeve arrives at the next gear, the impact forces cause the sleeve to bounce off the gear and delay the meshing process.}
%
\new{The number of bounces the sleeve encounters increases the order of overapproximation due to aggregation.}
%
The aggregation and de-aggregation techniques proposed in this paper are crucial for combating the computational cost while improving the accuracy for performing safety analysis.
%
%Due to the nondeterminism in the switching behavior, the aggregation and de-aggregation techniques proposed in this paper are crucial for the safety analysis.

%This paper exclusively focuses on improving the accuracy of the reachable set by presenting template based and convex hull based aggregation strategies for discrete transitions in linear hybrid automata.
%%
%We also present a data structure called Aggregated Directed Acyclic Graph (AGGDAG) and explain how aggdag helps us in implementing various deaggregation strategies. 
%%for keeping track of aggregations and their dependence on the reachable set.
%%
%%Using AGGDAG, we present several deaggregation strategies based on the counterexamples generated for the safety specification. 
%%
%We also provide soundness and relative completeness guarantees of our reachable set computation algorithm.
%%
%We implement our strategies in a tool called HyLAA~\cite{bak2017hscc} and analyze a satellite rendezvous mission.
%%
%Analyzing this model is particularly difficult because of the nondeterminism in switching behavior.
%%
%In the literature only restricted models of such switching behavior were analyzed owing to the overapproximation created by the state of the art tools.
%
%
%
%
%%
%The initial set of states after the mode switch is obtained by representing an overapproximation of the set of states that can 
%%
%Due to the non-determinism at a mode switch, the initial set of states for the next mode are obtained by performing 
%%
%When the executions of hybrid automata switch from one mode to another, one has to over
%%
%While performing such mode switches, one has to approximate a union of 
%%
%%
%Consider the case of a satellite rendezvous mission that involves a mode switch from continue to abort.
%%
%This mode switch can happen at any time in a given interval. 
%%
%Due to this non-determinism, the initial set of states in the abort mode is a convex overapproximation of the reachable set. 
%%
%Often, this overapproximation is too conservative and hence, such reachable set computation is not useful in determining whether all the behaviors of the systems are safe. 
%%
%While some approaches tried to avoid computing intersections, to compute the reachable set~\cite{althoff2012avoiding}, one has to perform overapproximation, which often becomes very conservative.
%
%\sridhar{Probably a spaceEx figure for state space exploration might come in handy here.}



%This work on developing deaggregation strategies is crucial for performing safety analysis of a satellite rendezvous mission. 
%%
%\sridhar{Might have to be ommitted.} Due to the non-determinism in the mode switching, state of the art tools are only able to analyze restricted switching behavior of this satellite rendezvous mission.
%%
%We demonstrate that using the deaggregation strategies presented in this paper, we can perform accurate safety analysis very quickly.
