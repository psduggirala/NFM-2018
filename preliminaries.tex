% !TEX root = emsoft-2019.tex
\section{Preliminaries}
\label{sec:prelims}
%\vspace{-0.2cm}

States and vectors, elements in $\reals^n$, are denoted as $x$ and $v$. 
%Given a sequence $seq = s_1, s_2, \ldots$, the $i^{th}$ element in the sequence is denoted as $seq[i]$. 
In this work, we use the following mathematical notation of a linear hybrid automata.

%\vspace{-0.1cm}
\begin{definition}
\label{def:hybridAutomata}
 A \emph{linear hybrid automaton} is defined to be a tuple
 $\tup{Loc, X, Flow, Inv, Trans, Guard, Reset}$ where:
% \vspace{-0.2cm}
\begin{description}
\item[$Loc$] is a finite set of locations (also called modes).
\item[$X$] $\subseteq \reals^n$ is the state space of the behaviors.
\item[$Flow$] $: Loc \rightarrow \textit{AffineDeq}(X)$ assigns an affine differential equation $\dot{x} = A_l x + B_l$ for location $l$ of the hybrid automaton.
\item[$Inv$] $: Loc \rightarrow 2^{X}$ assigns an invariant set for each location of the hybrid automaton.
\item[$Trans$] $\subseteq Loc \times Loc$ is the set of discrete transitions.
\item[$Guard$] $: Trans \rightarrow 2^{X}$ defines the set of states where a discrete transition is enabled.
\new{\item[$Reset$] $: Trans \times X \rightarrow X$ defines the reset function that determines the next state after the discrete transition.}
\end{description}

%\vspace{-0.2cm}
For a linear hybrid automaton, the invariants and guards are given as a conjunction of linear constraints and the reset function is an affine function.
\end{definition}
%\vspace{-0.1cm}

The \textit{set of initial states} $\Theta \deq (loc_0, S_0)$ where $loc_0 \in Loc$ is called the initial location and $S_0$ is given as a conjunction of linear constraints.
%
An \textit{initial state} $q_0$ is a pair $(Loc_0, x_0)$, such that $x_0 \in X$, and $(Loc_0, x_0) \in \Theta$.
%
Unsafe states $U$ is also given as a conjunction of linear constraints. 

\begin{definition}
\label{def:hybridExecution}
Given a hybrid automaton and an initial set of states $\Theta$, an \emph{execution} of the hybrid automaton is a sequence of trajectories and actions $\tau_0 a_1 \tau_1 a_2 \ldots $ such that
%
(i) the first state of $\tau_0$ denoted as $q_0$ is in the initial set, i.e., $q_0 = (Loc_0, x_0)\in \Theta$,
%
(ii) each $\tau_i$ is the solution of the differential equation of the corresponding location $Loc_i$, 
%
(iii) all the states in the trajectory $\tau_i$ respect the invariant of the location $Loc_i$,
%
and (iv) the state of the trajectory before each action $a_i$ satisfies $Guard(a_i)$.
\end{definition}
%
The set of states encountered by all executions that conform to the above semantics is called the \emph{reachable set}.
For linear systems, the closed form expression for the trajectories is given as $\tau_i(t) = e^{A_l t}\tau(0) + \int_{0}^{t}e^{A_l (t-\mu)}B_l d\mu$
where $A_l$ and $B_l$ define the affine dynamics of the mode.
%
Instead of computing the reachable set of states, we compute the set of states which can be reached by a fixed simulation algorithm.
%
We call this reachable set as \emph{simulation equivalent reachable set}, defined in~\cite{bak2017tacas}. We provide the details here for completeness.

\begin{definition}
\label{def:stepSim}
A sequence $\rho_{H}(q_0,h) = q_0, q_1, q_2, \ldots$, where each $q_i = (Loc_i, x_i)$, is a $(q_0, h)$-simulation of the hybrid automaton $H$ with initial set $\Theta$ if and only if $q_0 \in \Theta$ and each pair $(q_i, q_{i+1})$ corresponds to either: 
%
(i) a continuous trajectory in location $Loc_i$ with $Loc_i=Loc_{i+1}$ such that a trajectory starting from $x_i$ would reach $x_{i+1}$ after exactly $h$ time units with $x_i \in Inv(Loc_i)$, or 
%
(ii) a discrete transition from $Loc_i$ to $Loc_{i+1}$ (with $Loc_{i-1} = Loc_i$) where  $\exists a \in Trans$ such that $x_i = Reset(a, x_{i+1})$, $x_i \in Guard(a)$ and $x_{i+1} \in
Inv(Loc_{i+1})$.
%
Bounded-time variants of these simulations, with time bound $k\times h$, are called $(q_0, h, k)$-simulations.
\end{definition}

\begin{definition}[Simulation-Equivalent Reachable Set]
\label{def:simRS}
Given a hybrid automaton $H$, initial set $\Theta$, bounded time $T$, and simulation step size $h$, the simulation equivalent reachable set $RS$ is the set of all states $y$ such that there exists a simulation $\rho_{H}(q_0, h, k)$ with $q_0 \in \Theta$ that visits $y$.
\end{definition}

\begin{definition}[Simulation-Equivalent Safety]
\label{def:simSafe}
A hybrid automaton $H$ with initial set $\Theta$, time bound $T$, step size $h$, and unsafe set $U$ is said to be Simulation-equivalent safe, if all the simulations $\rho_{H}(q_0, h, k)$ with $q_0$ from $\Theta$ do not visit the unsafe set $U$.
\end{definition}

\new{\begin{remark}
Note that simulation-equivalent safety does not ensure that all executions (Definition~\ref{def:hybridExecution}) of the hybrid automaton are safe. The execution might encounter an unsafe state in between the time instances for performing safety verification. To the authors' knowledge, none of the tools available can provide relative completeness guarantees for safety property of executions. We stick to the notion of simulation equivalent safety because of two reasons. First, we can provide soundness and relative-completeness guarantees and second, for every unsafe system, our approach can generate a counterexample simulation. These guarantees come at a cost: the user should select the appropriate step size for analyzing the system under. In the case studies analyzed in this paper, we highlight the different choices of this step size and present the results.
\end{remark}}

\subsection{Symbolic Representation: Generalized Stars}
\label{sec:genStars}
We include the basic details of the symbolic representation called generalized stars~\cite{duggirala2016parsimonious,bak2017tacas} with some syntactic modifications. \new{Operations such as linear transformation, intersection, and Minkowski sum can be performed very efficiently over generalized stars, making them very suitable for reachable set computation.}
%For additional details, the readers can refer to~\cite{duggirala2016parsimonious,bak2017tacas}.

\begin{definition}
\label{def:genStar}
A \emph{generalized star} (or simply star) $\Theta$ is a tuple $\tup{a,G,P}$ where $a \in \reals^n$ is called the \emph{anchor}, $G = \{g_1,g_2,\ldots, g_n\}$ is a set of vectors in $\reals^n$ called the \emph{generators}, and $P: \reals^n \rightarrow \{\top, \bot\}$ is a predicate. A generalized star $\Theta$ defines a subset of $\reals^n$ as follows.
\begin{eqnarray*}
\means{\Theta} = \{ x\: |\: \exists \bar{\alpha} = [\alpha_1, \ldots, \alpha_n]^T \mbox{ such that }\\
 x = a + \Sigma_{i=1}^n \alpha_i g_i \mbox{ and } P(\bar{\alpha}) = \top \}
\end{eqnarray*}
Sometimes we will refer to both $\Theta$ and $\means{\Theta}$ as $\Theta$.
\end{definition}
The generalized stars that we encounter in our analysis have predicate $P$ defined as a conjunction of linear constraints.

\begin{example}
\label{ex:genStar}
A unit square $S$ in the cartesian plane with corners at $(0,0)$ and $(1,1)$ can be represented as a star $\tup{a, G, P}$ where $a = (0,0)$, $G = \{i, j\}$ where $i$ and $j$ are unit vectors along the $x$ and $y$ axes respectively, and $P \deq 0 \leq \alpha_1 \leq 1 \wedge 0 \leq \alpha_2 \leq 1$.

A set can be represeted in multiple ways in generalized star representation. Changing the anchor to $a' = (1,1)$ results in the new predicate $P' \deq -1 \leq \alpha_1 \leq 0 \wedge -1 \leq \alpha_2 \leq 0$. Changing the generators to $G' = \{i+j, i-j\}$ would result in the predicate $P'' \deq -\sqrt{2} \leq \alpha_1 + \alpha_2 \leq 0 \wedge -\sqrt{2} \leq \alpha_1 - \alpha_2 \leq 0$. Observe that changing the anchor would correspond to translation and changing the generators corresponds to a linear transformation over the predicate $P$.
\end{example}

%\subsubsection{Operations On Generalized Stars}

Given two generalized stars $\Theta_1 \deq \tup{a, G, P_1}$ and $\Theta_2 \deq \tup{a, G, P_2}$, $\Theta_1 \cap \Theta_2 \deq \tup{a, G, P_1 \wedge P_2}$. 
%
Here, the predicates $P_1$ and $P_2$ are defined over the same set of variables. 
%
If two stars do not share the same anchor and generators, one has to perform translation and linear transformation to have the same anchor and generators. 

%Given two generalized stars $\Theta \deq \tup{a, G, P}$ and $\Theta' \deq \tup{a', G', P'}$, the Minkowski sum of $\Theta \oplus \Theta' \deq \tup{a+a', G \cup G', P\wedge P'}$. Observe that the number of generators in $\Theta \oplus \Theta'$ is equal to the sum of number of generators in $\Theta$ and $\Theta'$. Additionally, we perform variable renaming such that the predicates $P$ and $P'$ do not share any variables.

\subsection{Reachable Set Computation of Linear Dynamical Systems Using Generalized Stars}
\label{sec:reachStars}

In this section we will outline the reachable set computation of linear dynamical systems that uses a symbolic representation called \emph{Generalized Stars}.
%
Generalized star representation leverages the superposition property of the linear dynamical systems~\cite{duggirala2016parsimonious}.
%

%\vspace{-0.7cm}
\vspace{-0.4cm}
\begin{algorithm}[h]
\SetAlgoVlined
\SetKwInOut{Input}{Input}\SetKwInOut{Output}{Output}\SetKw{Return}{return}
\Input{Initial Set: $\Theta = \tup{a, G, P}$, Dynamics: $(A, B)$, Step: $h$, Bound: $k$}
\Output{$Reach(\Theta) = Reach_0(\Theta), \ldots, Reach_k(\Theta)$}
\For{ each $i$ from $0$ to $k$}{
   $a_i \gets \rho(a, h, k)[i]$\; \label{ln:lineCenter}
   \For{ each $g_j \in V$ }{ \label{ln:begLoop}
      $g_j' \gets \rho(a+g_j, h, k)[i] - a_i$\; \label{ln:simVec}
   } \label{ln:endLoop}
   $G_i \gets \{ g_1', \ldots, g_m'\}$\;
   $Reach_i(\Theta) \gets \tup{a_i, G_i, P}$\; \label{ln:reachSeti1}
   Append $Reach_i(\Theta)$ to $Reach(\Theta)$\;
}
{\bf return} $Reach(\Theta)$\;
\caption{Algorithm that computes the reachable set for a linear dynamical system at time instances $i \cdot h$ from $n+1$ simulations.}
\label{alg:algoFullInfo}
\end{algorithm}
\vspace{-0.4cm}

Given an initial set $\Theta \deq \tup{a, G, P}$ with $G = \{g_1, g_2, \ldots, g_n\}$, we compute the reachable set for a linear dynamical system $\dot{x} = Ax + B$ using simulations.
%
We generate simulations starting from $a$ (denoted as $\rho(a, h, k)$), and $a+g_j$ $\forall 1\leq j \leq n$ (denoted as $\rho(a+g_j, h, k)$). 
%
For a given time instance $i\cdot h$, the reachable set denoted as $Reach_i(\Theta)$ is defined as $\tup{a_i, G_i, P}$ where $a_i = \rho(a, h, k)[i]$ and $G_i = \tup{g_1', g_2', \ldots, g_n'}$ where $\forall 1\leq j \leq n, g_j' = \rho(a+g_j, h, k)[i] - \rho(a, h, k)[i]$. 
%
Readers can refer to~\cite{bak2017tacas} for the correctness of this algorithm.
%
For ensuring that all trajectories starting from $\Theta$ satisfy the safety specification, one can check if there exists a set in $Reach(\Theta)$ overlapping with the unsafe set $U$.
%Notice that the predicate does not change for the reachable set, but only the center and the basis vectors are changed.
%

\begin{remark}
\label{rem:predConst}
Notice that the predicate in the reachable set remains same as that of the initial set $\Theta$. Hence, if one changes the initial set by changing the predicate, one needs not generate additional simulations for computing the reachable set. One can just change the predicate in line~\ref{ln:reachSeti1} and compute the new reachable set. For checking the safety specification with the new initial set, we still have to perform the overlap of the unsafe set $U$ with reachable set.
\end{remark}

\subsection{Reachable Set Computation of Linear Hybrid Systems Using Generalized Stars}
%Notice that Algorithm~\ref{alg:algoFullInfo} can compute the reachable set of linear dynamical systems using $n+1$ simulations. 
%
The reachable set computation technique for hybrid automata has three subroutines.
%
First, it computes the reachable set of the dynamical system in the current mode of operation (line~\ref{ln:reachDyn}).
%
Second, it computes the overlap of the reachable set in the current mode with the mode invariant (line~\ref{ln:overlapInv}).
%
Third, it computes the overlap of the reachable sets with the guards of discrete transitions that cause a change in mode (line~\ref{ln:guardIntersection1}).
%
When a discrete transition is performed, the reachable set in the new mode is computed by invoking the Algorithm~\ref{alg:algoFullInfo} subroutine.
%
Algorithm~\ref{alg:algoHybrid} is a pseudocode description of the algorithm. This reachable set computed is simulation equivalent reachable set, i.e., a state is in the reachable set if and only if there exists at least one simulation that visits the state.

\vspace{-0.4cm}
\begin{algorithm}[h!]
\SetAlgoVlined
\SetKwInOut{Input}{Input}\SetKwInOut{Output}{Output}\SetKw{Return}{return}
\Input{Initial set $\Theta$, Hybrid system $H$, Step: $h$, Bound: $k$.}
\Output{${\it ReachSet}$ as the set of reachable states.}
${\it queueStars} \gets \emptyset$\; 
append ($\Theta$, $loc_0$, $0$) to ${\it queueStars}$\; 
${\it ReachSet} \gets \emptyset$\;
\While{${\it queueStars}$ is not empty}{
    $S \gets {\sf dequeue}({\it queueStars})$\;
    $R \gets {\sf ReachSetDynSys}(S, S.loc, h, k-S.time)$\; \label{ln:reachDyn}
    $R' \gets {\sf InvariantOverlap}(R, R.Inv)$\; \label{ln:overlapInv}
    ${\it ReachSet} \gets {\it ReachSet} \cup R'$\; 
    ${\it nextRegions} \gets {\sf discreteTrans}(R', H.Trans)$\; \label{ln:guardIntersection1}
    append ${\it nextRegions}$ to ${\it queueStars}$\; \label{ln:appendQueueStars}
}
{\bf return} ${\it ReachSet}$\;
\caption{Algorithm that computes bounded time simulation equivalent reachable set.}
\label{alg:algoHybrid}
\end{algorithm}
\vspace{-0.4cm}

We want to highlight that the reachable set is only computed at discrete instances of time.
%
However, the advantage of our approach is that we can provide a counterexample when the safety specification is violated.
%
Algorithm~\ref{alg:algoHybrid} terminates because of two reasons. First, the simulations considered in this paper spend at least one step in its current mode of operation. Second, we compute the reachable set for only a bounded number of steps $k$. Hence, the set of states in ${\it queueStars}$ is finite.

One of the primary drawbacks of Algorithm~\ref{alg:algoHybrid} is in handling the discrete transitions. 
%
Suppose that in a given location, the number of stars that overlap with the guard of a discrete transition (line~\ref{ln:guardIntersection1} in Algorithm~\ref{alg:algoHybrid}) is $m$.
%
As a result, the number of stars in the $queueStars$ will become $O(m^2)$ after 2 discrete transitions. After $\eta$ number of discrete transitions, the number of states in $queueStars$ grows to $O(m^{\eta})$.
%
To avoid the exponential blow up of the number of sets in $queueStars$, reachable set computation tools often use aggregation.

\begin{remark}
\label{rem:hybridAlgo}
While Algorithm~\ref{alg:algoHybrid} might compute exponential number of stars, we do not generate exponential number of simulations for each mode in the hybrid automata. For each mode, we decide on an anchor $a_{mode}$ and generators $G_{mode}$ and pre-compute the $n+1$ simulations required for reachable set computation for that mode. For computing the reachable set of a star in a new mode (line~\ref{ln:reachDyn}), we change the anchor and the generators of the star to $a_{mode}$ and $G_{mode}$ respectively and use the observation in Remark~\ref{rem:predConst}. 

While this helps us cut down the number of simulations, we still have to check for overlap of unsafe set with exponential number of stars. Aggregation is crucial step to decrease the number of such possible overlaps.
\end{remark}


%\begin{remark}
%\label{rem:reachHybrid}
%
%\end{remark}
%
