\usepackage{tikz}
\usepackage[usenames,dvipsnames]{pstricks}
\usepackage{epsfig}

%\usepackage{palatino}
\RequirePackage{ifthen}
\RequirePackage{amsmath}
\RequirePackage{amssymb}
\RequirePackage{listings}
\RequirePackage{mathrsfs}
\RequirePackage{xspace}
\RequirePackage{graphics}

% PACKAGES FOR ORGANIZING THE STRUCTE OF DOCUMENT
%\usepackage{minipage}
\usepackage{latexsym}
\usepackage{multicol}
\usepackage{multirow}
\usepackage{lscape}  % Useful for wide tables or figures.
\usepackage{xspace}
\usepackage{paralist}
%\usepackage{framed}
%\usepackage{lipsum}
% DEFINING THE BOUNDARY --- DONE IN DOCUMENT.
%\usepackage[left=1.25in,right=1.25in,top=1in,bottom=1.25in]{geometry}

%%% SECTION TITLE APPEARANCE
% \usepackage{sectsty}
% \allsectionsfont{\sffamily\mdseries\upshape} % (See the fntguide.pdf for font help)
% (This matches ConTeXt defaults)

%%% HEADERS & FOOTERS
% \usepackage{fancyhdr} % This should be set AFTER setting up the page geometry
% \pagestyle{fancy} % options: empty , plain , fancy
% \renewcommand{\headrulewidth}{0pt} % customise the layout...
% \lhead{}\chead{}\rhead{}
% \lfoot{}\cfoot{\thepage}\rfoot{}


% PACKAGES FOR FIGURES
\usepackage{xcolor}
\usepackage{float}
\usepackage{subfigure}
%\usepackage{subfloat}
\usepackage{graphicx}
\usepackage{caption}
\usepackage{wrapfig}
%\usepackage{pifont}
%\usepackage{subcaption}
%\usepackage{subfigure}  % for subfigures
%\usepackage{setspace}
%\usepackage[justification=raggedright]{caption}	% makes captions ragged right - thanks to Bryce Lobdell

%AMS AND OTHER MISC PACKAGES
\usepackage{keyval}
\usepackage{amsmath,amssymb,amsfonts,url,listings,mathrsfs}
%reference
\usepackage{hyperref}
%\usepackage[pdftex,colorlinks,citecolor=webblue,linkcolor=webblue,urlcolor=webblue]{hyperref}

%ALGORITHM PACKAGE
%\usepackage{algorithm,algorithmic}
%
%\usepackage[pdftex,colorlinks,citecolor=webblue,linkcolor=webred,backref,pagebackref]{hyperref}
%\usepackage[linesnumbered,lined,commentsnumbered]{algorithm2e}
%\usepackage[linesnumbered,boxed,lined,commentsnumbered,algochapter]{algorithm2e}


%DEFINITIONS FOR MARKING CHANGES AND COMMENTS
\newcommand{\authcomment}[1]{\textbf{[[#1]]}}
%\newcommand{\sayan}[1]{\textcolor{blue}{[[#1]]}}
%\newcommand{\sridhar}[1]{\textcolor{red}{[[#1]]}}
\newcommand{\sridhar}[1]{\textcolor{brown}{[[#1]]}}
\newcommand{\stan}[1]{\textcolor{blue}{[[#1]]}}

% DEFINING COLORS
\definecolor{reddish}{rgb}{1,.8,0.8}
\definecolor{blueish}{rgb}{0.8,.8,1}
\definecolor{greenish}{rgb}{.8,1,0.8}
\definecolor{yellowish}{rgb}{1,1,.20}
\definecolor{webred}{rgb}{0.5,0,0}
\definecolor{webblue}{rgb}{0,0,0.8}

% DEFINITIONS FOR TEXT FORMATTING
\newcommand{\figuresize}{\scriptsize}
\newcommand{\equationsize}{\footnotesize}

% ADHOC DEFINITIONS FOR EACH AND EVERY PAPER
\newcommand{\trajmet}{discrepancy function\/}
\newcommand{\TrajMet}{Trajectory Metric}
\newcommand{\RT}{{\em reachtubes}}
\newcommand{\hylaa}{{HyLAA}}
\newcommand{\valsim}{{\sf valSim}}
\newcommand{\tool}{C2E2 \/}
\newcommand{\fulltoolname}{\textbf{C}heck \textbf{E}xecute \textbf{C}ompare \textbf{E}ngine}
\newcommand{\leftshift}[2]{\relax\ifmmode (#1 \lhd #2) \else $(#1 \lhd #2)$ \fi}
\newcommand{\mayInt}{\relax\ifmmode \mathit{mayInt} \else $\mathit{mayInt}$ \fi}
\newcommand{\mustInt}{\relax\ifmmode \mathit{mustInt} \else $\mathit{mustInt}$ \fi}
\newcommand{\notInt}{\relax\ifmmode \mathit{notInt} \else $\mathit{notInt}$ \fi}
\newcommand{\alarm}{\relax\ifmmode \mathit{Alert} \else $\mathit{Alert}$ \fi}
\newcommand{\unsafe}{\relax\ifmmode \mathit{Unsafe} \else $\mathit{Unsafe}$ \fi}
\newcommand{\xsep}{\relax\ifmmode \mathit{xsep} \else $\mathit{xsep}$\fi}
\newcommand{\ysep}{\relax\ifmmode \mathit{ysep} \else $\mathit{ysep}$ \fi}
\newcommand{\etime}{{\sf time}}
\newcommand{\ub}{{\sf ub}}
\newcommand{\lb}{{\sf lb}}
\newcommand{\emode}{{\sf mode}}
\newcommand{\erange}{{\sf range}}
\newcommand{\fbls} {functionally bounded linear span}
\newcommand{\fblsupper} {Functionally bounded linear span}

% COMMANDS USED IN DEFINITIONS
\newcommand{\meansl}{[\![}
\newcommand{\meansr}{]\!]}
\newcommand{\means}[1]{\meansl #1 \meansr}
\newcommand{\Nset}[1]{[[#1]]}

% ROUTINELY USED COMMANDS
\newcommand{\refinement}[1]{\relax\ifmmode \mathit{refine}(#1) \else $\mathit{refine}(#1)$ \fi}
\newcommand{\diff}{{\sf diff}}
\newcommand{\diameter} {\mathsf{diameter}}
\newcommand{\cntr} {\mathsf{center}}
\newcommand{\ball}[1] {B_{#1}}
\newcommand{\true}{\relax\ifmmode \mathit true \else \em true \/\fi}
\newcommand{\false}{\relax\ifmmode \mathit false \else \em false \/\fi}
% \newcommand{\lnot}{\neg} % Already defined
% \newcommand{\land}{\wedge} % Already defined
% \newcommand{\lor}{\vee} % Already defined
\newcommand{\limplies}{\Rightarrow}
\newcommand{\liff}{\Leftrightarrow}
%\newcommand{\T}{\mathit{true}}
%\newcommand{\F}{\mathit{false}}
% \newcommand{\to}{\relax\ifmmode \rightarrow \else $\rightarrow$\fi} % Already defined
\newcommand{\fro}{\relax\ifmmode \leftarrow \else $\leftarrow$\fi}


% COMMANDS FOR MATH SPECIFIC ITEMS
\newcommand{\nnt}{{\sf T}^{\geq 0}}             %nonnegative time points
\newcommand{\post}{{\sf T}^{>0}}                %positive time points
\newcommand{\Variables}{{\sf V}}                %variables

\newcommand{\num}[1]{\relax\ifmmode \mathbb #1\else $\mathbb #1$\fi}
\newcommand{\nnnum}[1]{\relax\ifmmode 
  {\mathbb #1}_{\geq 0} \else ${\mathbb #1}_{\geq 0}$
  \fi}
\newcommand{\npnum}[1]{\relax\ifmmode 
  {\mathbb #1}_{\leq 0} \else ${\mathbb #1}_{\leq 0}$
  \fi}
\newcommand{\pnum}[1]{\relax\ifmmode 
  {\mathbb #1}_{> 0} \else ${\mathbb #1}_{> 0}$
  \fi}
\newcommand{\nnum}[1]{\relax\ifmmode 
  {\mathbb #1}_{< 0} \else ${\mathbb #1}_{< 0}$
  \fi}
\newcommand{\plnum}[1]{\relax\ifmmode 
  {\mathbb #1}_{+} \else ${\mathbb #1}_{+}$
  \fi}
\newcommand{\nenum}[1]{\relax\ifmmode 
  {\mathbb #1}_{-} \else ${\mathbb #1}_{-}$
  \fi}

\newcommand{\reals}{{\num R}}                    %reals
\newcommand{\nnreals}{{\nnnum R}}                    %nonnegative reals
\newcommand{\realsinfty}{{\num R} \cup \{\infty, -\infty\}}                    %nonnegative reals
\newcommand{\plreals}{{\plnum R}}                    %positive reals
\newcommand{\naturals}{{\num N}}                      %natural numbers
\newcommand{\integers}{{\num Z}}                      %integers
\newcommand{\rationals}{{\num Q}}                      %rationals
\newcommand{\nnrationals}{{\nnnum Q}}                   % nonnegative rationals
\newcommand{\Time}{{\num T}}  
\newcommand{\bools}{{\num B}}  
\newcommand{\plintegers}{{\plnum Z}}                      %integers
\newcommand{\deq}{\mathrel{\stackrel{\scriptscriptstyle\Delta}{=}}}
\newcommand{\intervals}{\relax \ifmmode \mathbb{IR} \else $\mathbb{IR}$ \fi}
\newcommand{\nnintervals}{\relax \ifmmode \mathbb{IR}^{+} \else $\mathbb{IR}^{+}$ \fi}
%\setlength{\mathindent}{0pt}

% COMMANDS FOR LABELS
\renewcommand{\eqref}[1]{Equation~(\ref{eq:#1})}
\newcommand{\eqlabel}[1]{\label{eq:#1}}
\newcommand{\figref}[1]{Figure~\ref{fig:#1}}
\newcommand{\figrefs}[2]{Figures~\ref{fig:#1} and~\ref{fig:#2}}
\newcommand{\figlabel}[1]{\label{fig:#1}}
\newcommand{\tabref}[1]{Table~\ref{table:#1}}
\newcommand{\tablabel}[1]{\label{table:#1}}
\newcommand{\defref}[1]{Definition~\ref{def:#1}}
\newcommand{\deflabel}[1]{\label{def:#1}}
\newcommand{\exref}[1]{Example~\ref{exmp:#1}}
\newcommand{\exlabel}[1]{\label{exmp:#1}}
\newcommand{\scref}[1]{Section~\ref{sec:#1}}
\newcommand{\secreftwo}[2]{Sections~\ref{sec:#1}~and~\ref{sec:#2}}
\newcommand{\sclabel}[1]{\label{sec:#1}}
\newcommand{\applabel}[1]{\label{app:#1}}
\newcommand{\appref}[1]{Appendix~\ref{app:#1}}
\newcommand{\lnlabel}[1]{\label{line:#1}}
\newcommand{\lnrngref}[2]{lines~\ref{line:#1}--\ref{line:#2}\xspace}
\newcommand{\lnref}[1]{line~\ref{line:#1}\xspace}
\newcommand{\thmref}[1]{Theorem~\ref{thm:#1}\xspace}
%\newcommand{\seclabel}[1]{\label{sec:#1}}
%\newcommand{\secref}[1]{Section~\ref{sec:#1}}
%\newcommand{\figlabel}[1]{\label{fig:#1}}
%\newcommand{\figref}[1]{Figure~\ref{fig:#1}}


% COMMANDS FOR SECTIONS AND ORGANIZATION

\newenvironment{noqedproof}{\pf}{}
\newcommand{\pf}{\par\noindent{\bf Proof:}~}
% \newcounter{theorem}
% \setcounter{theorem}{0}
% 
%  \newtheorem{theorem}{Theorem}
%  \newtheorem{lemma}[theorem]{Lemma}
%  \newtheorem{corollary}[theorem]{Corollary}
%  \newtheorem*{claim}{Claim}
%  \newtheorem{assumption}{Assumption}
%  \theoremstyle{definition}
%  \newtheorem{definition}{Definition}
%  \newcommand{\qed}{\hfill{\rule{2mm}{2mm}}\medskip}
%  \newenvironment{proof}{\pf}{\qed}
%  \newtheorem{proposition}[theorem]{Proposition}
% \newtheorem{inv}[theorem]{Invariant}
 
%    \newcounter{rem}
%  \setcounter{rem}{0}
%  
%   \newenvironment{remark}
%   {\refstepcounter{rem} \vspace{2ex}\par\noindent
%    \textbf{Remark}~\textbf{\therem~}}
 
% \newcounter{example}
%  \setcounter{example}{0}

%   \newtheorem{example}[example]{Example}
%  \newcounter{example}
  
%  \newenvironment{example}
%  {\refstepcounter{example} \vspace{2ex}\par\noindent
%  \textbf{Example}~\textbf{\theexample~}}%{\qed}
% \newexample{example}[example]{Example}

%\theoremstyle{remark}
%\newcounter{example}
% \newtheorem{example}{Example}[chapter]
% {\refstepcounter{theorem} \vspace{2ex}\par\noindent
% \textbf{Example}~\textbf{\theexampletheorem~}}{\qed}
 
 \def\examplenonum#1#2{
        \vspace{.15in}
        \noindent
        {\bf Example #1}{\em ~(continued)}{\bf .}
        {#2}{\qed}
        }
        
% \renewcommand{\theequation}{\thesection.\arabic{equation}}
% \renewcommand{\thefigure}{\thesection.\arabic{figure}}
% \renewcommand{\thetable}{\thesection.\arabic{table}}
% \newcommand{\Section}[1]{\section{#1}%
%    \setcounter{equation}{0}\setcounter{figure}{0}\setcounter{table}{0}%
%    \setcounter{example}{0}}


%%%%%%%%%%%%%%%%%
%%%%%%%%%%%%%%%%%
%%%%% TIOA STUFF %%%%%
%%%%%%%%%%%%%%%%%
%%%%%%%%%%%%%%%%%



% EXECUTIONS TRACES and FRAGS
\newcommand{\extb}[1]{\relax\ifmmode {\sf ExtBeh}_{#1} \else ${\sf ExtBeh}_{#1}$\fi} 
\newcommand{\tdists}[1]{\relax\ifmmode {\sf Tdists}_{#1} \else ${\sf Tdists}_{#1}$\fi} 

\newcommand{\exec}[1]{\relax\ifmmode {\sf Execs}_{#1} \else ${\sf Exec}_{#1}$\fi} 
\newcommand{\execf}[1]{\relax\ifmmode {\sf Execs}^*_{#1} \else ${\sf Exec}^*_{#1}$\fi} 
\newcommand{\execi}[1]{\relax\ifmmode {\sf Execs}^\omega_{#1} \else ${\sf Exec}^\omega_{#1}$\fi} 

\newcommand{\ctrace}[1]{\relax\ifmmode {\sf Ctraces}_{#1} \else ${\sf Ctraces}_{#1}$\fi} 

\newcommand{\trace}[1]{\relax\ifmmode {\sf Traces}_{#1} \else ${\sf Traces}_{#1}$\fi} 
\newcommand{\tracef}[1]{\relax\ifmmode {\sf Traces}^*_{#1} \else ${\sf Traces}^*_{#1}$\fi} 
\newcommand{\tracei}[1]{\relax\ifmmode {\sf Traces}^\omega_{#1} \else ${\sf Traces}^\omega_{#1}$\fi} 

\newcommand{\frag}[1]{\relax\ifmmode {\sf Frags}_{#1} \else ${\sf Frags}_{#1}$\fi} 
\newcommand{\fragf}[1]{\relax\ifmmode {\sf Frags}^*_{#1} \else ${\sf Frags}^*_{#1}$\fi} 
\newcommand{\fragi}[1]{\relax\ifmmode {\sf Frags}^\omega_{#1} \else ${\sf Frags}^\omega_{#1}$\fi} 

\newcommand{\reach}[1]{\relax\ifmmode {\sf Reach}_{#1} \else ${\sf Reach}_{#1}$\fi} 
\newcommand{\pair}[2]{\relax\ifmmode \langle #1, #2 \rangle \else $\langle #1, #2 \rangle$\fi} 

\newcommand{\TE}{\relax\ifmmode \mathit{Time} \else $\mathit{Time}$ \fi} 
\newcommand{\EQ}{\relax\ifmmode \mathit{Enq} \else $\mathit{Enq}$ \fi} 
\newcommand{\DQ}{\relax\ifmmode \mathit{Deq} \else $\mathit{DeqTime}$ \fi} 
\newcommand{\E}{\relax\ifmmode \mathsf{E} \else $\mathsf{E}$ \fi}

\newcommand{\loc}{\relax\ifmmode \mathit{loc} \else $\mathit{loc}$ \fi}
\newcommand{\abs}{\relax\ifmmode \mathit{abs} \else $\mathit{abs}$ \fi}

\newcommand{\execs}{{\exec{}}}
\newcommand{\traces}{{\trace{}}}
\newcommand{\fragss}{{\frag{}}}
\newcommand{\fexecs}{{\execf{}}}
\newcommand{\ftraces}{{\tracef{}}}
\newcommand{\ffragss}{{\fragf{}}}
\newcommand{\iexecs}{{\execi{}}}
\newcommand{\itraces}{{\tracei{}}}
\newcommand{\ifragss}{{\fragi{}}}
\newcommand{\fstate}{{\sf fstate}}  
\newcommand{\lstate}{{\sf lstate}}  
\newcommand{\ltime}{{\sf ltime}}  
\newcommand{\dur}{{\sf dur}}  
\newcommand{\mode}{{\sf mode}}  
\newcommand{\ftime}{{\sf ftime}}  

% 
% \newenvironment{anotation}[1][Nancy]{\begin{quote}\small[[[#1]]]}{\normalsize\end{quote}}
% \newcommand{\ba}{\begin{anotation}}
% \newcommand{\ea}{\end{anotation}}
% 
% \newcommand{\bcb}{\chgbarbegin}
% \newcommand{\ecb}{\chgbarend}
% \chgbarwidth 1pt
% 
% \newcommand{\dec}{\ensuremath{{:}}}
% \newcommand{\eqdef}{\mathbin{::=}}
% \newcommand{\I}{{\ensuremath{\cap}}}



% OPERATIONS ON SETS, RELATIONS AND FUNCTIONS
\newcommand{\pow}[1]{{\bf P}(#1)} % powerset
\newcommand{\inverse}[1]{#1^{-1}}
\newcommand{\range}[1]{\ms{range{(#1)}}}
\newcommand{\domain}[1]{{\it dom}(#1)}
\newcommand{\type}[1]{\ms{type{(#1)}}}
\newcommand{\dtype}[1]{\ms{dtype{(#1)}}} % dynamic type
\newcommand{\restr}{\mathrel{\lceil}}
\newcommand{\proj}{\matrel{\lceil}}
\newcommand{\restrrange}{\mathrel{\downarrow}}
\newcommand{\point}[1]{\wp(#1)}                 %point trajectory



% HYBRID AUTOMATA
\def\A{{\cal A}} % HA
\def\B{{\cal B}} % HA
\def\C{{\cal C}} % HA
\def\D{{\cal D}} % set of discrete steps
\def\E{{\cal E}} % HA
\def\F{{\cal F}} % HA
\def\G{{\cal G}} % pieces of SHIOA
\def\H{{\cal H}} % HA
\def\I{{\cal I}} % environment sequence
\def\K{{\cal K}} % environment sequence
\def\L{{\cal L}} % environment sequence
\def\M{{\cal M}} % Mode switching transitions
\def\O{{\cal O}} % outcome function
\def\P{{\cal P}} % set of modes
\def\Q{{\cal Q}} % set of modes
\def\R{{\cal R}} % relation
\def\S{{\cal S}} % set of trajectories
\def\T{{\cal T}} % set of trajectories
\def\V{{\cal V}} % Lyapunov function
\def\U{{\cal U}} % set of trajectories
\def\X{{\cal X}} % Lyapunov function
\def\Y{{\cal Y}} % set of trajectories
\def\Z{{\cal Z}} % set of trajectories

\def\u{{\mathsf u}} % set of inputs

% more special characters

\newcommand{\col}[1]{\relax\ifmmode \mathscr #1\else $\mathscr #1$\fi}
\def\statemodels{\col{S}}


% Names of actions, automata etc
\definecolor{HIOAcolor}{rgb}{0.776,0.22,0.07}
\newcommand{\HIOA}{\textcolor{HIOAcolor}{\tt HIOA\hspace{3pt}}}
\newcommand{\PVS}{\textcolor{HIOAcolor}{\tt PVS\hspace{3pt}}}
\newcommand{\PVSnogap}{\textcolor{HIOAcolor}{\tt PVS\hspace{1pt}}}
\newcommand{\HIOAbiggap}{\textcolor{HIOAcolor}{\tt HIOA\hspace{6pt}}}
\newcommand{\HIOAnogap}{\textcolor{HIOAcolor}{\tt HIOA}}
\newcommand{\anyrelation}{\lessgtr}

% Transformation for ADT
\newcommand{\SC}[2]{\relax\ifmmode {\tt Scount}(#1,#2) \else ${\tt Scount}(#1,#2)$\fi} 
\newcommand{\SCM}[2]{\relax\ifmmode {\tt Smin}(#1,#2) \else ${\tt Smin}(#1,#2)$\fi} 
\newcommand{\Aut}[1]{\relax\ifmmode {\tt Aut}(#1) \else ${\tt Aut}(#1)$\fi} 

\newcommand{\auto}[1]{{\operatorname{\mathsf{#1}}}}
\newcommand{\act}[1]{{\operatorname{\mathsf{#1}}}}
\newcommand{\smodel}[1]{{\operatorname{\mathsf{#1}}}}
\newcommand{\pvstheory}[1]{{\operatorname{\mathit{#1}}}}
%\newcommand{\auto}[1]{\relax\ifmmode \sf #1\else $\sf #1$\fi}
%\newcommand{\act}[1]{\relax\ifmmode \sf #1\else $\sf #1$\fi}


\newcommand{\Automaton}{{\bf automaton}}
\newcommand{\Asserts}{{\bf asserts}}
\newcommand{\Assumes}{{\bf assumes}}
\newcommand{\Backward}{{\bf backward}}
\newcommand{\By}{{\bf by}}
\newcommand{\Case}{{\bf case}}
\newcommand{\Choose}{{\bf  choose}}
\newcommand{\Components}{{\bf components}}
\newcommand{\Const}{{\bf const}}
\newcommand{\Converts}{{\bf converts}}
\newcommand{\Do}{{\bf do}}
\newcommand{\Eff}{{\bf eff}}
\newcommand{\Else}{{\bf else}}
\newcommand{\Elseif}{{\bf elseif}}
\newcommand{\Enumeration}{{\bf enumeration}}
\newcommand{\Ensuring}{{\bf ensuring}}
\newcommand{\Exempting}{{\bf exempting}}
\newcommand{\Fi}{{\bf fi}}
\newcommand{\For}{{\bf for}}
\newcommand{\Forward}{{\bf forward}}
\newcommand{\Freely}{{\bf freely}}
\newcommand{\From}{{\bf from}}
\newcommand{\Generated}{{\bf generated}}
\newcommand{\Local}{{\bf local}}
\newcommand{\Hidden}{{\bf hidden}}
\newcommand{\If}{{\bf if}}
\newcommand{\In}{{\bf in}}
\newcommand{\Implies}{{\bf implies}}
\newcommand{\Includes}{{\bf includes}}
\newcommand{\Introduces}{{\bf introduces}}
\newcommand{\Input}{{\bf input}}
\newcommand{\Kind}{{\bf kind}}
\newcommand{\Initially}{{\bf initially}}
\newcommand{\Internal}{{\bf internal}}
\newcommand{\Invariant}{{\bf invariant}}
\newcommand{\Od}{{\bf od}}
\newcommand{\Of}{{\bf of}}
\newcommand{\Output}{{\bf output}}
\newcommand{\Partitioned}{{\bf partitioned}}
\newcommand{\Pre}{{\bf pre}}
\newcommand{\Signature}{{\bf signature}}
\newcommand{\Simulation}{{\bf simulation}}
\newcommand{\Sort}{{\bf sort}}
\newcommand{\States}{{\bf states}}
\newcommand{\Tasks}{{\bf tasks}}
\newcommand{\Then}{{\bf then}}
\newcommand{\To}{{\bf to}}
\newcommand{\Trait}{{\bf trait}}
\newcommand{\Traits}{{\bf traits}}
\newcommand{\Transitions}{{\bf transitions}}
\newcommand{\Tuple}{{\bf tuple}}
\newcommand{\Type}{{\bf type}}
\newcommand{\Union}{{\bf union}}
\newcommand{\Uses}{{\bf uses}}
\newcommand{\Where}{{\bf where}}
\newcommand{\While}{{\bf while}}
\newcommand{\With}{{\bf with}}

% Spacing
\newcommand{\FFF}{\vspace{0.1in}}
\newcommand{\BBB}{\hspace{-0.1in}}

