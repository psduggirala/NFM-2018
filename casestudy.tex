\section{Case Study: Spacecraft Rendezvous Passive Safety}
\label{sec:casestudy}

We evaluate our method on a spacecraft rendezvous passive safety case study. The system consists of a primary chaser spacecraft moving towards a secondary, free-flying object (such as a satellite) and performing close-proximity maneuvers. The verification goal is to ensure \emph{passive safety}: at any time in the maneuver, a system failure may occur and the resulting propulsion-free trajectory must avoid colliding with the target satellite. This requirement comes from real-world failures. In 2005, NASA’s DART spacecraft was intended to rendezvous with the MUBLCOM satellite, but due to depleted propellant instead collided with the target satellite (a loss of a
\$110 million project)~\cite{croomes2006overview}.

Our model is based on a published benchmark for this system~\cite{chan2017verifying,jewison2016spacecraft}. The system is modeled as a hybrid automaton with different discrete modes depending on the sensors being used for navigation, and an LQR controller is designed to meet physical and geometric safety constraints. The relative dynamics are linearized using the Clohessy-Wiltshire-Hill (CWH) equations~\cite{wh1960terminal}, which is often used in close proximity operations. The hybrid automaton consists of three modes, two for different navigation strategies, and one for the passive dynamics.

Although several tools have successfully analyzed a version of this model in the ARCH hybrid systems tools competition in 2018~\cite{archcomp18linear}, a critical simplification was made: the competition model did not actually check the passive safety requirement. In particular, the competition model used a \emph{fixed} time to transition to the passive model. This is an unrealistic simplification, since the time of failure cannot not be known in advance. 

The analysis done in the original work~\cite{chan2017verifying} is slightly better, in that it checks for passive safety for a known 5 minute failure interval during the 200 minute maneuver. The reason stated in the paper for this is that, if larger time intervals are used, “the initial set of states under the Passive mode is large, making it very difficult to prove safety.” The suggestion is then to create subintervals that cover the full time range of transitions to the passive mode, and then run several experiments. Presumably, a manual guess-and-check approach should be used to create these subintervals.

The full passive safety problem can be solved using the proposed aggdag method. Our aggdag method performs full state aggregation (an overapproximation), and then recursively desegregates if the overapproximation reaches an error mode. The advantage of this is that (1) the method is fully automatic, (2) steps where the overapproximation is safe can be skipped by the refined sets, which is more efficient than using multiple independent experiments, and (3) if an error exists, it will be detected after full deaggregation is performed, which allows the generation of a concrete counter example trace. Other verification tools for linear hybrid systems do not typically generate counter-examples when safety cannot be proven.

